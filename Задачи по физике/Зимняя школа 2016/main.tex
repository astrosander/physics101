\documentclass[12pt,a4paper]{article}
%\usepackage{cmap}
%\usepackage[T1, T2A]{fontenc}
\usepackage[utf8]{inputenc}
\usepackage[russian]{babel}
\usepackage{amsmath,amssymb}
\usepackage{parskip}
\usepackage{caption}
\usepackage{textcomp}
\usepackage{gensymb}
\usepackage[dvips]{graphicx}
\usepackage{wrapfig}
\usepackage{color}
\usepackage{setspace}
%\usepackage{hyperref}
\usepackage{epstopdf}

\oddsidemargin=0 cm
\evensidemargin=0 cm
\textwidth=170 mm
\textheight=260 mm
\topmargin=0 cm
\voffset= -2cm
\pagenumbering{false}
%\newlength{\varheight}
%\setlength{\varheight}{3.1cm}
\setlength{\parindent}{0cm}
\spacing{1.1}
\parskip=2mm
\clubpenalty=10000
\widowpenalty=10000

\begin{document}

\begin{center}
\LARGE{Задачи на зимнюю школу}
\end{center}
\vspace{6mm}

{\large{Бесконечные цепи}}

[1]. Способ подсчета сопротивления (самоподобие, ``метод Ийона Тихого''). Простые примеры: $R=(1+\sqrt5)/2$, $R=(1+\sqrt{4R_2/R_1})/2$.

2. Винница-16, теор старшей, задача 2. Симметрия, сворачивание цепи пополам.

$$r=\cdot\frac{1+\sqrt{21}}{5+\sqrt{21}}$$

3. Допы (всеросс), задача 3.52. Суммирование двух рядов по отдельности, сумма геометрической прогрессии, расходимость гармонического ряда. $r_1=-1+\sqrt{17}/2$, $r_2=-3/2+\sqrt{17}/2$.

[4]. $LC$-цепочка, условие возникновения волны, дисперсионное соотношение, скорость волны. Комплексные импедансы.

Дисперсионное соотношение
$$
  \omega(\lambda)=\frac{2}{\sqrt{LC}}\sin\frac{\pi l}{\lambda},
$$
волна не выживает при $\omega>2/\sqrt{LC}$. Скорость $u=\omega \lambda$.

{\large{Проволочные сетки}}

[1]. Простейшая квадратная сетка, сопротивление между соседними. Независимость источника и стока, принцип суперпозиции, симметрия. $r=1/2$.

[2]. Треугольная сетка, между соседними. $r=1/3$.

3. Допы (всеросс), задача 3.53, треугольная ``через одно'', выразить одно через другое. $r_1=2R-r/2$.

4. Квадратная, между вершиной и серединой ребра. $r=3/8$.

{\large{Приближенный счет сопротивлений}}

1. Высокоомный участок в низкоомной цепи (201.09.19). Последовательный расчет потенциалов (высокоомное включение не меняет распределение потенциалов в низкоомной цепи). $U=2$ V.

[2]. Допы (всеросс), задача 3.50, та же идея. $i\approx \varepsilon/(18000R)$.

{\large{Сила Архимеда и силы давления воды}}

1. Винница-16, теор старшей. Расчет силы давления на плоскую грань, сила Архимеда как сумма сил давления.

$$
  F_2=\sqrt{F_1^2+F_\text{A}^2}=\rho gS\sqrt{H^2+\frac{h^2}{9}}.
$$

{\large{Максимизация заполнением изооболочек}}

1. Задача с планетой, заполнение оболочек с максимальным $dg/dm$.

$$
  \frac{z}{\left(r^2+z^2\right)^{3/2}}=const
$$

{\large{Кумулятивные шарики}}

1. Эстонско-финская 2014, задача 4. Случай легких шариков, решение простой рекурренты. Для сравнимых по массе: по формуле.

{\large{Задачи с построениями и измерениями на графиках}}

1. Оптика, эстонско-финская, 2006, задача 2. На догадку: использовать шкалу линейки на изображении, пятна --- изображения точечных источников.

2. Эстонско-финская 2007, задача 5. Построение окружности по 3 точкам, разложение на поступательное и вращательное движения.

3. Эстонско-финская 2008, задача 3. Трактора, измерение проекций.

{\large{Уравнения Максвелла}}

1. Поле полусферы на срезе. Симметрия, уравнение $\text{curl } E=0$.

{\large{Висячая нить}}

1. Отборы на межнар, Гельфгат 2012, задача 2. Интегрирование вдоль нити, преобразование дифференциалов.

[2]. Тот же отбор, задача 3. Форма нитей, натяжение не вертикально. Трансцендентное уравнение решать не обязательно.

3. Всеукр 2013, 11 класс, задача 4. Интегрирование вдоль нити и по координатным осям.

{\large{Термодинамика и графики}}

1. Областная 2015, 11 класс, задача 4. Подсчет работы как площади.

\end{document}

