\documentclass[12pt,a4paper,pdflatex]{disser}
%\usepackage[russian]{babel}
%\usepackage[utf-8]{inputenc}
%\usepackage{amsmath,amssymb}
%\usepackage{longtable}
\usepackage{parskip}
\usepackage{caption}
\usepackage{textcomp}
\usepackage{gensymb}
\usepackage[dvips]{graphicx}
%\usepackage{wrapfig}
%\usepackage{amssymb}*
%\usepackage{color}
%\usepackage{ulem}
\usepackage{setspace}
\usepackage{hyperref}

\oddsidemargin=0 cm
\evensidemargin=0 cm
\textwidth=170 mm
\textheight=260 mm
\topmargin=0 cm
\voffset= -2cm
\pagenumbering{true}
%\newlength{\varheight}
%\setlength{\varheight}{3.1cm}
\setlength{\parindent}{0cm}
%\newcommand{\taskname}[name]{\begin{center} \bf{\Large{name}} \end{center}}
\spacing{1.1}
\parskip=2mm
%\captionsetup[figure]{labelformat=empty}
\clubpenalty=10000
\widowpenalty=10000

\begin{document}

1. Consider a current $i$ flowing through the straight wire. It creates a magnetic field
$$
  B(r)=\frac{\mu_0 i}{2\pi r}.
$$
Then the flux through the circular loop is
$$
  \Phi=\int\limits_0^{2a} B(r)\sqrt{a^2-(a-r)^2}dr=\frac{\mu_0 i}{2\pi}\int\limits_0^{2a} \sqrt{\frac{2a-r}{r}}dr=\frac{\mu_0 ia}{2}.
$$
This yields the mutual inductance
$$
  M=\frac{\Phi}{i}=\frac{\mu_0 a}{2}.
$$

2. The magnetic field separates the charges on the sphere by the Lorentz force, as a consequence the charges create a force field equivalent to an electric field $E=Bv$ in the $-z$-direction. Then the surface charge density depends on the angle $\varphi$ between the $z$ axis and the radius-vector by law
$$
  \sigma(\varphi)=-3\varepsilon_0 E\cos\varphi=-3\varepsilon_0 Bv\cos\varphi.
$$

3. The distribution of the magnetic field induces a current in the loop which dissipates the gravitational energy. The total resistance of the loop is
$$
  R=\frac{4\rho D}{d^2},
$$
and the mass is
$$
  m=\frac{\pi^2 d^2 D\rho_m}{4}.
$$
The induced emf is
$$
  \mathcal{E}=\frac{d\Phi}{dt}=\frac{\pi D^2 \kappa v}{4},
$$
where $v$ is the velocity of the loop. Then the power of dissipation is
$$
  P=\frac{\mathcal{E}^2}{R}=\frac{\pi^2 d^2 D^3 \kappa^2 v^2}{64\rho}.
$$
For a steady velocity $P=mgv$, which yields
$$
  v=\frac{16\rho \rho_m g}{\kappa^2 D^2}.
$$

4. Integrating the wave function gives
$$
  A=2\alpha^{3/4}\left(\frac{2}{\pi}\right)^{1/4}.
$$
Substituting the given functions $\psi(x)$ and $U(x)$ into the differential equation, we get
$$
  \alpha=\frac{m\omega}{2\hbar}
$$
and
$$
  \varepsilon=\frac{3\hbar\omega}{2}.
$$
Several integrations yield
\begin{gather*}
  \langle x\rangle=\int\limits_{-\infty}^\infty x\psi^2(x)dx=0,\\
  \langle x^2\rangle=\int\limits_{-\infty}^\infty x^2\psi^2(x)dx=\frac{3A^2}{16\alpha^{5/2}}\sqrt{\frac{\pi}{2}}=\frac{3}{4\alpha},\\
  \langle p\rangle=-i\hbar\int\limits_{-\infty}^\infty \psi(x)\psi'(x)dx=0,\\
  \langle p^2\rangle=\hbar^2\int\limits_{-\infty}^\infty \psi(x)\psi''(x)dx=\frac{3A^2}{4}\sqrt{\frac{\pi}{2\alpha}}=3\hbar^2 \alpha.\\
\end{gather*}
Then
\begin{gather*}
  \Delta x=\sqrt{\langle x^2\rangle-\langle x\rangle^2}=\frac{1}{2}\sqrt{3}{\alpha},\\
  \Delta p=\sqrt{\langle p^2\rangle-\langle p\rangle^2}=\hbar\sqrt{3\alpha}.
\end{gather*}
In this case
$$
  \Delta p\cdot\Delta x=\frac{3\hbar}{2}\ge\frac{\hbar}{2}.
$$

\end{document} 