\documentclass[14pt,a4paper,pdflatex]{disser}
%\usepackage[russian]{babel}
%\usepackage[utf-8]{inputenc}
%\usepackage{amsmath,amssymb}
%\usepackage{longtable}
\usepackage{parskip}
\usepackage{caption}
\usepackage{textcomp}
\usepackage{gensymb}
\usepackage[dvips]{graphicx}
%\usepackage{wrapfig}
%\usepackage{amssymb}*
%\usepackage{color}
%\usepackage{ulem}
\usepackage{setspace}
\usepackage{hyperref}

\oddsidemargin=0 cm
\evensidemargin=0 cm
\textwidth=170 mm
\textheight=260 mm
\topmargin=0 cm
\voffset= -2cm
\pagenumbering{true}
%\newlength{\varheight}
%\setlength{\varheight}{3.1cm}
\setlength{\parindent}{0cm}
%\newcommand{\taskname}[name]{\begin{center} \bf{\Large{name}} \end{center}}
\spacing{1.1}
\parskip=2mm
%\captionsetup[figure]{labelformat=empty}
\clubpenalty=10000
\widowpenalty=10000

\begin{document}

1. Integration gives
$$
  B=\frac{\mu_0 I}{2\pi}\left(1-\frac{a+d}{\sqrt{a^2+d^2}}\right),
$$
the positive direction is by default downwards (away from the picture).

2. Consider a loop current $i_1$ in the bigger loop. It creates a field
$$
  B_0=\frac{\mu_0 i_1}{2R_1}
$$
in its center. In case of $R_1\gg R_2$ we can consider $B_0$ as a uniform field containing the smaller coil. Then the magnetic flux is
$$
  \Phi_2=\pi R_2^2 B_0=\pi R_2^2 \cdot \frac{\mu_0 i_1}{2R_1},
$$
then the mutual inductance is
$$
  M=\frac{\mu_0 \pi R_2^2}{2R_1}.
$$

3. The general formula is
$$
  1-\beta^2=\left(\frac{mc^2}{E}\right)^2.
$$
As $E\gg mc^2$ (for an electron $mc^2=512$ keV), then a small quantity
$$
  \varepsilon=1-\beta=\frac{1}{2}\left(\frac{mc^2}{E}\right)^2=3.6\cdot 10^{-9},
$$
and the electron's velocity is
$$
  \beta=1-\varepsilon=0.9999999964.
$$

4. The rockets' relative velocity is
$$
  \beta=\frac{\beta_1+\beta_2}{1+\beta_1 \beta_2}=0.8.
$$
Then the contraction factor is $l_0/l=\gamma=\left(1-\beta^2\right)^{-1/2}=5/3$.

5. The momentum of the system is conserved and is equal to
$$
  p=\frac{m_0 v}{\sqrt{1-v^2/c^2}}.
$$
At the end the velocity of the system is
$$
  \beta_1=\left(1+\left(\frac{2m_0 c}{p}\right)^2\right)^{-1/2}=\left(\frac{4c^2}{v^2}-3\right)^{-1/2},
$$
so the mass of the system is
$$
  m_1=\frac{p}{\beta_1 c}=m_0\sqrt{\frac{4-3\beta_0^2}{1-\beta_0^2}},
$$
where $\beta_0$ corresponds to $v/c$.

6. Consider a photon with momentum $\overrightarrow{p}$ creating an electron-positron pair with momenta $\overrightarrow{p_1}$ and $\overrightarrow{p_2}$ respectively. Then due to conservation laws
$$
  \overrightarrow{p}=\overrightarrow{p_1}+\overrightarrow{p_2},
$$
$$
  pc=\sqrt{p_1^2 c^2+m^2 c^4}+\sqrt{p_2^2 c_2+m^2 c^4},
$$
where $m$ corresponds to the electron's (positron's) mass. Squaring both these equations and eliminating $p^2 c^2$, we get
$$
  \overrightarrow{p_1}\cdot \overrightarrow{p_2}=m^2 c^2+\sqrt{\left(p_1^2+m^2 c^2\right)\left(p_2^2+m^2 c^2\right)}.
$$
This can't correspond to the reality as $\overrightarrow{p_1}\cdot \overrightarrow{p_2}\le p_1 p_2<\sqrt{\left(p_1^2+m^2 c^2\right)\left(p_2^2+m^2 c^2\right)}.$

\end{document} 