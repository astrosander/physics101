\documentclass[14pt]{article}

\usepackage[russian]{babel}
\usepackage[utf8]{inputenc}
\usepackage{amsmath,amssymb}
\usepackage{parskip}
\usepackage{caption}
\usepackage{textcomp}
\usepackage{gensymb}
\usepackage[dvips]{graphicx}
\usepackage{wrapfig}
\usepackage{color}
\usepackage{setspace}
%\usepackage{hyperref}
\usepackage{epstopdf}

\oddsidemargin=0 cm
\evensidemargin=0 cm
\textwidth=170 mm
\textheight=230 mm
\topmargin=0 cm
\voffset= -2cm
\pagenumbering{false}
\newlength{\varheight}
\setlength{\varheight}{3.1cm}
\setlength{\parindent}{0cm}
\spacing{1.1}
\parskip=2mm
\clubpenalty=10000
\widowpenalty=10000
\captionsetup[figure]{labelformat=empty}

\begin{document}

\begin{center}
\Large{\textbf{Системы координат - 1}}

\textbf{18.04.2017}

\vspace{5mm}
\end{center}

\textit{Retake version. No partial credit will be given for any problem!}

1. \textit{(0.25 каждая)} Переведите из декартовых координат в полярные: $(-5,5\sqrt3)$; $(-24,-7)$; $(\sqrt3,-\sqrt3)$; $(12,5)$; $(-5\sqrt2,5\sqrt2)$; $(-7/\sqrt3,-7)$; $(4,-3)$; $(0,6)$; $(-5,-10)$; $(-10\sqrt3,5\sqrt2)$; $(-6,0)$; $(-3\sqrt3,3)$; $(10,1)$; $(-1,-\sqrt3)$; $(-3,-6)$; $(5\sqrt3,-5\sqrt6)$; $(-10,15)$; $(1,-1)$; $(-9,40)$; $(-9,3\sqrt3)$. Изобразите все эти точки на рисунке.

2. \textit{(0.25 каждая)} Переведите из полярных координат в декартовы: $(10,-5\pi/3)$; $(3,-\pi)$; $(4,17\pi/4)$; $(2\sqrt2,0)$; $(41,-\pi-\arctan(40/9))$; $(15,76\pi/3)$; $(5,7\pi/2-\arctan(4/3))$; $(12,-95\pi/6)$; $(3,-7\pi/6)$; $(13,11\pi/2+\arccos(5/13))$, $(17,12\pi)$, $(25,-\pi+\arctan(24/7))$, $(9,-29\pi/3)$, $(4,19\pi/6)$, $(6,17\pi/2)$, $(17,-9\pi-\arccos(8/17))$, $(4,25\pi/4)$, $(29,15\pi-\arctan(20/21))$, $(18,-7\pi/3)$, $(5,11\pi/6)$. Изобразите все эти точки на рисунке.

3. \textit{(1.00 каждый)} Постройте графики линий, задаваемых в полярных координатах уравнениями:

а) $r(\varphi)=1+\dfrac{\varphi}{3\pi}$, $0<\varphi<7\pi$;
\hspace{3.5cm}
б) $r(\varphi)=\dfrac{1}{1-\sin\varphi}$, $0<\varphi<2\pi$;

в) $r(\varphi)=1+0.3\sin 3\varphi$, $0<\varphi<2\pi$;
\hspace{2.55cm}
г) $r(\varphi)=\dfrac{1}{1+\cos\varphi}$, $0<\varphi<2\pi$;

д) $r(\varphi)=2-0.5\sin \dfrac{\varphi}{2}$, $0<\varphi<4\pi$.

4. \textit{(0.25 каждая)} Переведите из декартовых координат в цилиндрические: $(-20,-21,e/2)$; $(3,-\sqrt3,-\pi)$; $(5,-3,4)$; $(6\sqrt2, 2\sqrt6,-6\sqrt2)$; $(-\sqrt6,\sqrt6,\sqrt2)$; $(\sqrt5,-\sqrt5,11\pi/4)$; $(-3,3\sqrt3,2)$; $(-7,24,13)$; $(0,6\sqrt2,-\pi/3)$; $(12, 10, -3)$; $(-9,-40,\pi)$; $(1,-2,1)$; $(-3,4,5)$; $(-2,-8,0)$; $(-7\sqrt3,7,-\pi/2)$; $(2,2,-2)$; $(6,-6\sqrt2,3)$; $(-8,-9,10)$; $(-1,1,-1)$; $(-41,8,20)$.

5. \textit{(0.25 каждая)} Переведите из цилиндрических координат в декартовы: $(2,7\pi/6,\pi)$; $(10,-5\pi/3,-1)$; $(5,9\pi/2-\arctan(3/4))$; $(12,-5\pi/6,0)$; $(7,\pi,-1)$; $(25,-5\pi/2-\arcsin(24/25),-3\pi/2)$; $(13,\arccos(-5/13),4)$; $(\sqrt14,-5\pi/4,\sqrt7)$; $(41,-\arccos(-9/41),-5)$; $(5,3\pi,\pi/4)$; $(13,\arctan(-12/5),7)$; $(125,\arcsin(-24/25),-e-1)$; $(8,-43\pi/4,\pi/7)$; $(13,\arccos(-5/13),-1)$; $(5,-\pi/4,4)$; $(17,-\pi/2+\arccos(-16/17),4)$; $(14,11\pi/6,-3)$; $(2,\pi/3,0)$; $(13,-17\pi/2+\arctan(-12/5),4)$; $(17,\-9\pi,\pi/2)$.

6. \textit{(0.5 каждая)} Переведите из декартовых координат в сферические: $(1,1,-1)$; $(-2,2,-1)$; $(-3,-4,-12)$; $(0,-1,\sqrt3)$; $(3,-4,12)$; $(4\sqrt2,4\sqrt2,-8\sqrt3)$; $(-10\sqrt3,10,-21)$; $(-6,0,6\sqrt3)$; $(3,4,5)$; $(-20\sqrt3,20,-9)$.

7. \textit{(0.5 каждая)} Переведите из сферических координат в декартовы: $(1,\pi/2,0)$, $(4,\pi/4,5\pi/3)$; $(2,3\pi/4,\pi)$; $(29,\arccos(-21/29),\arctan(-3/4))$; $(13,-\arctan(12/13),\pi/2+\arcsin(3/5))$; $(6,5\pi/6,-3\pi/4)$; $(10,0,\pi)$;\newline$(21,\arcsin(19/21),-\pi/6)$; $(10,\pi/4,7\pi/6)$; $(15,\pi/2,7\pi/4)$.

8. Получите формулы для перехода из цилиндрических координат в сферические и наоборот. С их помощью переведите:

а) \textit{(0.25 каждая)} из цилиндрических координат в сферические: $(3,3\pi/7,-4)$; $(2,7\pi/6,2)$; $(12,e/2,-5)$; $(3,12\pi/11,3\sqrt3)$; $(1,6\pi/5,-1/\sqrt3)$; $(5,\pi/3,5\sqrt2)$; $(9\sqrt3,5\pi/6,-9)$; $(1,0,0)$; $(3,\pi/2,3)$; $(1,\pi/2,\pi)$.

б) \textit{(0.25 каждая)} из сферических координат в цилиндрические: $(\sqrt2,3\pi/4,7)$; $(4,\pi/2,-\pi/6)$; $(2,5\pi/6,2)$; $(1,0.01,\pi)$; $(6,\pi/4,0)$; $(13,\pi-\arccos(5/13),\pi/3)$; $(5,\pi/2+\arctan(4/3),-2)$; $(1,\pi/7,e/3)$; $(1,\pi/2,0)$; $(4,5\pi/12,0)$.

\end{document} 