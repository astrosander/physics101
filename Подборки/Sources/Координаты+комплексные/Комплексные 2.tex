\documentclass[14pt]{article}

\usepackage[russian]{babel}
\usepackage[utf8]{inputenc}
\usepackage{amsmath,amssymb}
\usepackage{parskip}
\usepackage{caption}
\usepackage{textcomp}
\usepackage{gensymb}
\usepackage[dvips]{graphicx}
\usepackage{wrapfig}
\usepackage{color}
\usepackage{setspace}
%\usepackage{hyperref}
\usepackage{epstopdf}

\oddsidemargin=0 cm
\evensidemargin=0 cm
\textwidth=170 mm
\textheight=230 mm
\topmargin=0 cm
\voffset= -2cm
\pagenumbering{false}
\newlength{\varheight}
\setlength{\varheight}{3.1cm}
\setlength{\parindent}{0cm}
\spacing{1.1}
\parskip=2mm
\clubpenalty=10000
\widowpenalty=10000
\captionsetup[figure]{labelformat=empty}

\begin{document}

\begin{center}
\Large{\textbf{Комплексные числа - 2}}

\textbf{19.04.2017}
\vspace{5mm}
\end{center}

1. \textit{(0.25 каждое)} Представьте числа в экспоненциальной форме: $-7$; $4-3i$; $i-1$; $5i-12$; $20-21i$; $i$; $-3i-\sqrt3$; $1-i$; $-i-\sqrt2$; $-11i$.

2. \textit{(0.25 каждое)} Представьте числа в стандартной форме: $6e^{-3i\pi}$; $5e^{i\arctan(-4/3)}$; $10e^{37i\pi/6}$; $e^{-11i\pi/4}$; $8e^{4i\pi/3}$; $13e^{-47i\pi/6}$; $29e^{-3i\pi/4}$; $7e^{-i\pi}$; $13e^{i(\pi/2+\arccos(12/13))}$; $5e^{i(\pi-\arctan(3/4))}$.

3. 1) \textit{(0.50)} Покажите, что если при целом $n$ выполнено равенство $a^n=z$, то также выполнено равенство $\left(ae^{2\pi ik/n}\right)^n=z$ для любого целого $k$.

2) \textit{(1.00)} Найдите все возможные значения корня $\sqrt[\leftroot{-2}\uproot{2}n]{z}$ для натуральных $n$, если одно из них равно $a$. Сколько среди них различных? Изобразите их на графике при $z=i$ и $n=6$.

3) \textit{(0.50)} Покажите, что если выполнено равенство $e^a=z$, то также выполнено равенство $e^{a+2\pi ik}=z$ для любого целого $k$.

4) \textit{(1.00)} Найдите все возможные значения логарифма $\ln z$, если одно из них равно $a$. Сколько среди них различных? Изобразите 5 из них на графике при $z=-e$ и при $z=i$.

4. \textit{(0.25 каждое)} Найдите:

\begin{tabular}{llll}
1) $i^{229}$, \hspace{1.5cm} & 2) $(-i)^{4851}$, \hspace{1.5cm} & 3) $i^{101}+i^{102}+i^{103}$, \hspace{1.2cm} & 4) $i^{-413}$,\\
5) $i^{225}-i^{224}-i^{-224}+i^{-225}$, & 6) $(i^{253}+i^{250})i^{-343}$, & 7) $(-3i)^{-20}$ & 8)$(i^{-20}+(-i)^{-21})i^3$,\\
9) $(1-i)^{51}$, & 10) $(i\sqrt3-1)^{20}$, & 11) $(3+4i)^{5\pi/\arctan(4/3)}$ & 12)$\left(\dfrac{i+1}{\sqrt2}\right)^{-12}$,\\
13) $\text{Re}\,(29e^{i(7\pi/2-\arctan(20/21))})$, & 14) $\text{Im}\,|9-7i|$, & 15) $\text{Re}\,(5e^{\pi-\arctan(4/3)})$ & 16) $\text{Im}\,(i^{228}+(1+i)^{14})$,\\
17) $\left|\dfrac{-i-\sqrt3}{2}\right|^{25}$, & 18) $\left|3i+4\right|^{16}$, & 19) $\left|1+i\right|^{-13}$ & 20) $\left|i^{3204}\right|$.
\end{tabular}

\end{document} 