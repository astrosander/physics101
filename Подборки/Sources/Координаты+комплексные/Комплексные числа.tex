\documentclass[14pt]{article}

\usepackage[russian]{babel}
\usepackage[utf8]{inputenc}
\usepackage{amsmath,amssymb}
\usepackage{parskip}
\usepackage{caption}
\usepackage{textcomp}
\usepackage{gensymb}
\usepackage[dvips]{graphicx}
\usepackage{wrapfig}
\usepackage{color}
\usepackage{setspace}
%\usepackage{hyperref}
\usepackage{epstopdf}

\oddsidemargin=0 cm
\evensidemargin=0 cm
\textwidth=170 mm
\textheight=230 mm
\topmargin=0 cm
\voffset= -2cm
\pagenumbering{false}
\newlength{\varheight}
\setlength{\varheight}{3.1cm}
\setlength{\parindent}{0cm}
\spacing{1.1}
\parskip=2mm
\clubpenalty=10000
\widowpenalty=10000
\captionsetup[figure]{labelformat=empty}

\begin{document}

\begin{center}
\Large{\textbf{Комплексные числа - 1}}

\textbf{14.04.2017}

\vspace{5mm}
\end{center}

Декартова (стандартная) форма комплексного числа $z=x+iy$, экспоненциальная форма $z=\rho e^{i\varphi}$.

1. а) Какому вектору на плоскости соответствует комплексное число $z=x+iy$, число $z=\rho e^{i\varphi}$?

б) Покажите, что операциям сложения комплексных чисел и умножения комплексного числа на действительное соответствуют сложение векторов на плоскости и умножение вектора на число.

в$^*$) Какая функция комплексных чисел $z_1$, $z_2$ соответствует скалярному произведению векторов $\vec{z}_1\cdot \vec{z}_2$?

2. Представьте число в экспоненциальной форме: $3$; $3+4i$; $-4-3i$; $2-i$; $-4$; $i-1$.

3. Представьте число в стандартной форме: $2e^{i\pi}$; $10e^{-i \pi/2}$; $5e^{i\arctan(3/4)}$; $10e^{7\pi i/6}$; $e^{0i}$; $e^{5i\pi/4}$; $3e^{17\pi/3}$; $2e^{-29\pi/6}$; $10e^{11\pi/4}$.

4. По какой траектории движется точка, координаты которой в зависимости от времени $t$ задаются комплексным числом $z(t)$:

а) $v_0 t$, где $v_0$ --- постоянное комплексное число,

б) $r_0 e^{i\omega t}$, где $r_0$, $\omega$ --- положительные числа,

в) $t+it^2/t_0$, где $t_0$ --- положительное число,

г$^*$) $\omega Rt+iR+R e^{-i\omega t}$, где $\omega$, $R$ --- положительные числа?

5. Упростите:

\begin{tabular}{llll}
а) $(1+i)^3$, \hspace{2cm} & б) $(1+2i)(3-i)$, \hspace{2cm} & в) $(-1-i)^{100}$, \hspace{2cm} & г) $i^{2055}$,\\
д) $\left(\dfrac{i-1}{\sqrt2}\right)^{228}$, & е) $(i+\sqrt3)^{25}$, & ж) $(-2i)^{47}$ & з$^*$) $\dfrac{2-i}{1+i}$.
\end{tabular}

6. Найдите:

\begin{tabular}{llll}
а) $|e^{-ix}|$, \hspace{2cm} & б) $\arg(i-1)$, \hspace{2cm} & в) $\text{Re}\,(-4e^{3i\pi/4})$, \hspace{2cm} & г) $\arg\left((1-i\sqrt3)^{283}\right)$,\\
д) $\text{Im}\,(e^{-11\pi/6})$, & е) $\arg(re^{i\omega t})$, & ж) $\left|(1+i)^{102}\left(\dfrac{i+\sqrt3}{2}\right)^{73}\right|$ & з) $\text{Re}\,(10i-7)$.
\end{tabular}

7. Запишите в комплексном виде векторы скорости и ускорения частицы, если координаты задаются числом $z(t)$:

\begin{tabular}{lll}
а) $re^{i\omega t}$, \hspace{4cm} & б) $r_0+v_0 t+at^2/2$, \hspace{4cm} & в) $\omega Rt+iR+R e^{-i\omega t}$.
\end{tabular}

\end{document} 