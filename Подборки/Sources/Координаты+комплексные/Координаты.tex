\documentclass[14pt]{article}

\usepackage[russian]{babel}
\usepackage[utf8]{inputenc}
\usepackage{amsmath,amssymb}
\usepackage{parskip}
\usepackage{caption}
\usepackage{textcomp}
\usepackage{gensymb}
\usepackage[dvips]{graphicx}
\usepackage{wrapfig}
\usepackage{color}
\usepackage{setspace}
%\usepackage{hyperref}
\usepackage{epstopdf}

\oddsidemargin=0 cm
\evensidemargin=0 cm
\textwidth=170 mm
\textheight=230 mm
\topmargin=0 cm
\voffset= -2cm
\pagenumbering{false}
\newlength{\varheight}
\setlength{\varheight}{3.1cm}
\setlength{\parindent}{0cm}
\spacing{1.1}
\parskip=2mm
\clubpenalty=10000
\widowpenalty=10000
\captionsetup[figure]{labelformat=empty}

\begin{document}

\begin{center}
\Large{\textbf{Системы координат - 1}}

\textbf{14.04.2017}

\vspace{5mm}
\end{center}

2D: декартовы координаты $(x,y)$, полярные координаты $(\rho,\varphi)$.

3D: декартовы координаты $(x,y,z)$, цилиндрические координаты $(\rho,\varphi,z)$, сферические координаты $(r,\theta,\varphi)$.

1. а) Введите определение полярных координат, изобразите их на рисунке.

б) Получите формулы для перехода из декартовых координат в полярные и наоборот.

в) Что представляют собой линии, задаваемые уравнениями $\rho=\text{const}$ и $\varphi=\text{const}$? Изобразите их на рисунке.

2. Переведите из декартовых координат в полярные: $(0,1)$; $(2,-2)$; $(-3,0)$; $(-3,4)$; $(10,10)$; $(-1,-\sqrt3)$; $(0,0)^*$. Изобразите все эти точки на рисунке.

3. Переведите из полярных координат в декартовы: $(1,\pi/4)$; $(5,\pi)$; $(3,7\pi/4)$; $(4,0)$; $(2,-\pi/2)$; $(5,\pi+\arctan(4/3))$; $(13,2\pi-\arccos(12/13))$; $(9,34\pi/3)$; $(10,-73\pi/6)$; $(1,41\pi/4)$. Изобразите все эти точки на рисунке.

4. Постройте графики линий, задаваемых в полярных координатах уравнениями:

а) $r(\varphi)=1+\dfrac{\varphi}{4\pi}$, $0<\varphi<10\pi$;
\hspace{3cm}
б) $r(\varphi)=1+0.1\sin 3\varphi$, $0<\varphi<2\pi$;

в) $r(\varphi)=\dfrac{1}{1-0.5\cos\varphi}$, $0<\varphi<2\pi$.

5. а) Введите определение цилиндрических координат, изобразите их на рисунке.

б) Получите формулы для перехода из декартовых координат в цилиндрические и наоборот.

в) Что представляют собой поверхности, задаваемые уравнениями $\rho=\text{const}$, $\varphi=\text{const}$ и $z=\text{const}$?

6. Переведите из декартовых координат в цилиндрические: $(1,0,-2)$; $(-1,1,0)$; $(3,-3,3)$; $(-\sqrt3,1,2)$; $(-\sqrt6,0,-4)$; $(0,0,1)^*$.

7. Переведите из цилиндрических координат в декартовы: $(2,4\pi/3,-1)$; $(7\sqrt2,-\pi/4,0)$; $(1,-5\pi/6,\pi)$; $(2,0,-1)$; $(5,\pi+\arcsin(4/5),0)$; $(25,7\pi+\arccos(7/24),-4\pi)$; $(25,\pi/2,3)$.

8. а) Введите определение сферических координат, изобразите их на рисунке.

б) Получите формулы для перехода из декартовых координат в сферические и наоборот.

в) Что представляют собой поверхности, задаваемые уравнениями $\rho=\text{const}$, $\theta=\text{const}$ и $\varphi=\text{const}$?

9. Переведите из декартовых координат в сферические: $(1,1,1)$; $(-2,0,2)$; $(-1,-1,-1)$; $(1,0,-1)$; $(0,-3,4)$; $(5,0,-5\sqrt3)$; $(3,-4,-12)$; $(0,0,5)^*$; $(0,0,0)^*$.

10. Переведите из сферических координат в декартовы: $(1,0,0)$, $(13,\arctan(5/12),\pi-\arcsin(3/5))$; $(2,\pi/2,3\pi/4)$; $(1,2\pi/3,0)$; $(5,0,\pi/2)$; $(4,\pi/3,\pi/3)$.

\end{document} 