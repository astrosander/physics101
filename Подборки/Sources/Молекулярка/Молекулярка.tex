\documentclass[14pt]{article}

\usepackage[russian]{babel}
\usepackage[utf8]{inputenc}
\usepackage{amsmath,amssymb}
\usepackage{parskip}
\usepackage{caption}
\usepackage{textcomp}
\usepackage{gensymb}
\usepackage[dvips]{graphicx}
\usepackage{wrapfig}
\usepackage{color}
\usepackage{setspace}
%\usepackage{hyperref}
\usepackage{epstopdf}

\oddsidemargin=0 cm
\evensidemargin=0 cm
\textwidth=170 mm
\textheight=230 mm
\topmargin=0 cm
\voffset= -2cm
\pagenumbering{false}
\newlength{\varheight}
\setlength{\varheight}{3.1cm}
\setlength{\parindent}{0cm}
\spacing{1.1}
\parskip=2mm
\clubpenalty=10000
\widowpenalty=10000
\captionsetup[figure]{labelformat=empty}

\begin{document}

\begin{center}
\Large{\textbf{Бесконечные цепи и сетки}}

\textbf{18.04.2017}

\vspace{5mm}
\end{center}
{\large{Молекулярно-кинетическая теория}}
\vspace{3mm}

1. Пусть задана случайная величина $\xi$, причем вероятность ее нахождения в интервале $[x,x+dx]$ равняется $f(x)dx$. Функцию $f(x)$ называют функцией распределения величины $\xi$. Так, распределение молекул идеального газа по $x$-проекциям скоростей имеет вид
$$
  f_x(v_x)dv_x=A\exp\left(-\frac{m v_x^2}{2kT}\right)dv_x,
$$
где $m$ --- масса молекулы газа, $T$ --- его температура, $R$ --- газовая постоянная. Распределение молекул по проекциям скоростей на другие координатные оси имеет тот же вид.

а) Найдите $A$ из условия нормировки, т.е. из того, что вероятность попадания $v_x$ в интервал $[-\infty,+\infty]$ равна 1. Ответ выразите через $m$, $k$, $T$.

б) Пространством скоростей называют такое пространство, на осях которого отложены не координаты, а скорости $v_x$, $v_y$, $v_z$ молекулы. Укажите, в какой области этого пространства находятся молекулы, модуль скорости которых лежит в интервале $[v,v+dv]$.

в) Вероятность попадания молекулы в элементарный объем $[v_x,v_x+dv_x]\times[v_y,v_y+dv_y]\times[v_z,v_z+dv_z]$ имеет вид $f(v_x,v_y,v_z)dv_x dv_y dv_z$. Найдите $f(v_x,v_y,v_z)$.

г) Используя результаты предыдущих двух пунктов, найдите функцию распределения $f(v)$ идеального газа по модулям скоростей, т.е. вероятность попадания модуля скорости в интервал $[v,v+dv]$, деленную на $dv$.

д) Среднеквадратичная скорость молекул газа определяется соотношением $\langle v^2 \rangle=\int\limits_{-\infty}^\infty v^2f(v)dv$, средний модуль скорости --- соотношением $\langle v \rangle=\int\limits_{-\infty}^\infty vf(v)dv$. Найдите $\langle v\rangle$ и $\langle v^2 \rangle$.

е) Кинетическая энергия одной молекулы газа $E=mv^2/2$. Усредняя, получим среднюю кинетическую энергию $\langle E \rangle=m\langle v^2\rangle/2$. Найдите $\langle E \rangle$.

2. Количество ударов молекул идеального газа о стенки сосуда на единицу площади стенок в единицу времени равно
$$
  J=\frac{d^2 N}{dSdt}=\frac{n\langle v\rangle}{4},
$$
где $n$ --- его концентрация, $\langle v\rangle$ --- средний модуль скорости молекул. Рассмотрим орбитальную станцию массой $M=10000$ т и внутренним объемом $V=10^4$ м$^3$. Внутри станции поддерживается постоянная температура $T=300$ К и давление $p=10^5$ Па. Пусть в некоторый момент времени астероид пробил в корпусе станции дырку площадью $S=2$ см$^2$. Найдите:

а) зависимость концентрации газа в станции от времени $n(t)$,

б) скорость $v$, приобретенную станцией из-за утечки за время $t=500$ с. Считайте, что молекулы газа вылетают из дырки равномерно по всем направлениям.


\vspace{5mm}
{\large{Процессы в атмосфере}}
\vspace{3mm}

9. Пусть температура на поверхности Земли $T_0$, давление $p_0$. Молярная масса воздуха $\mu$, универсальная газовая постоянная $R$. Выразите через эти величины давление $p(z)$ на высоте $z$, если температура атмосферы зависит от высоты по закону

а) $T(z)=T_0$ (изотермическая атмосфера),

б) $T(z)=T_0-kz$.

Поднятие влажного воздуха и выпадение осадков: IPhO 1987-1

Адиабатическая конвекция, потолок конвекции и конвекционная устойчивость: IPhO 2008-3

10. (Эстонско-финская, 2008-7) В безветренную погоду на земле горит костер, дым от которого на высоте $h=7$ м имеет температуру $T=40\degree$C. Температуру атмосферы считайте постоянной, не зависящей от высоты и равной $T_0=20\degree$C. Атмосферное давление на поверхности Земли также постоянно и равно $p_0=100$ кПа. Дым считайте идеальным газом с молярной массой $\mu=0.029$ кг/моль и теплоемкостью при постоянном объеме $c_V=5R/2$ ($R=8.31$ Дж/(моль$\cdot$К) --- газовая постоянная). На какую максимальную высоту поднимется столб дыма?

\end{document} 