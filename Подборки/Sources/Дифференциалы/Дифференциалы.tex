\documentclass[14pt]{article}

\usepackage[russian]{babel}
\usepackage[utf8]{inputenc}
\usepackage{amsmath,amssymb}
\usepackage{parskip}
\usepackage{caption}
\usepackage{textcomp}
\usepackage{gensymb}
\usepackage[dvips]{graphicx}
\usepackage{wrapfig}
\usepackage{color}
\usepackage{setspace}
%\usepackage{hyperref}
\usepackage{epstopdf}

\oddsidemargin=0 cm
\evensidemargin=0 cm
\textwidth=170 mm
\textheight=230 mm
\topmargin=0 cm
\voffset= -2cm
\pagenumbering{false}
\newlength{\varheight}
\setlength{\varheight}{3.1cm}
\setlength{\parindent}{0cm}
\spacing{1.1}
\parskip=2mm
\clubpenalty=10000
\widowpenalty=10000
\captionsetup[figure]{labelformat=empty}

\begin{document}

\begin{center}
\Large{\textbf{Полные дифференциалы}}

\textbf{23.04.2017}

\vspace{5mm}
\end{center}

Приращением функции $f(t)$ называют величину $\Delta f=f(t+dt)-f(t)$. Если $f=f(t)$, то полным дифференциалом называют величину $df=f'(t)dt$, по сравнению с приращением по формуле Тейлора
$$
  \Delta f=f(t+dt)-f(t)=f'(t)dt+\dfrac12 f''(t)dt^2+\dfrac16 f^{(3)}(t)dt^3+\ldots
$$
В большинстве задач нужны только дифференциалы первого порядка, поэтому приращение величины $\Delta f$ за время $dt$ равно $df=f'(t)dt$. Дифференциалы можно и нужно понимать как бесконечно малые приращения. Для функции нескольких переменных $f(x,y,z,\ldots)$
$$
  df=\frac{\partial f}{\partial x}dx+\frac{\partial f}{\partial y}dy+\frac{\partial f}{\partial z}dz+\ldots
$$
Величина $(\partial f/\partial x)dx$ есть приращение функции $f$ при изменении только аргумента $x$. Соответственно, полное приращение есть сумма таких величин для разных аргументов.

1. Запишите полные дифференциалы следующих величин:

\begin{tabular}{llll}
1) $\cos x$, \hspace{1.5cm} & 2) $\tan x$, \hspace{1.5cm} & 3) $\dfrac{1}{x^3}$, \hspace{1.5cm} & 4) $x+y$,\\
5) $i^{225}-i^{224}-i^{-224}+i^{-225}$, & 6) $(i^{253}+i^{250})i^{-343}$, & 7) $(-3i)^{-20}$ & 8)$(i^{-20}+(-i)^{-21})i^3$,\\
9) $(1-i)^{51}$, & 10) $(i\sqrt3-1)^{20}$, & 11) $(3+4i)^{5\pi/\arctan(4/3)}$ & 12)$\left(\dfrac{i+1}{\sqrt2}\right)^{-12}$,\\
13) $\text{Re}\,(29e^{i(7\pi/2-\arctan(20/21))})$, & 14) $\text{Im}\,|9-7i|$, & 15) $\text{Re}\,(5e^{\pi-\arctan(4/3)})$ & 16) $\text{Im}\,(i^{228}+(1+i)^{14})$,\\
17) $\left|\dfrac{-i-\sqrt3}{2}\right|^{25}$, & 18) $\left|3i+4\right|^{16}$, & 19) $\left|1+i\right|^{-13}$ & 20) $\left|i^{3204}\right|$.
\end{tabular}

\end{document} 